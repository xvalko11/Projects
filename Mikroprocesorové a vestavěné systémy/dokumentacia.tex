\documentclass[a4paper, 11pt]{article}
\usepackage[utf8]{inputenc}
\usepackage[left=2cm, top=3cm, text={17cm, 24cm}]{geometry}
\usepackage{graphicx}
\usepackage[slovak]{babel}
\usepackage{times}
\usepackage[unicode]{hyperref}
\usepackage{bookmark}
\usepackage{amssymb}
\usepackage{enumitem}
\begin{document}

\thispagestyle{empty}
\begin{center}
\begin{figure}[h]
\centering
\scalebox{0.9}{\includegraphics{fit banner.png}}
\end{figure}
\vspace{\stretch{0.382}}
\LARGE Dokumentácia projektu z predmetu IMP\\
\LARGE \textbf{Hra HAD}\\
\LARGE ARM-FITkit3\\
\vspace{\stretch{0.618}}
\Large \hfill Marek Valko (xvalko11)\\
\pagebreak
\end{center}

\tableofcontents
\pagebreak

\section{Úvod do problematiky}
\subsection{O hre HAD}
Had (červ) je bežný názov pre 2D hry vytvorené na základe jednoduchého hern0ho designu s obmezenou hernou plochou, v rámci ktorej sa had pohybuje. Herné skóre rastie spoločne s predlužujúcou se dĺžkou hada na základe zjedenej potravy nachádzajúcej sa vo vnútri hernej plochy. Hadia hra končí v momente nárazu do prekážky - hlavou do vlastného tela hada, do okraja hernej plochy prípadne iného hada.

Herný design hry had sa datuje desiatky rokov naspäť, do novembra 1976, kedy Gremlin Industries predstavili arkádovú videohru Blockade. O rok neskôr (1977), svetlo sveta uzreli 2 nové hry inšpirované hrou Blockade, vytvorené spoločnosťou Atari, určené pre hernú konzolu Atari VCS. Prvá verzia hry Had určená pro osobní počítače (TRS-80, Commodore PET, Apple II) bola vydaná v roku 1978 Petrm Trefonasem, nasledovaná ďalšími verziami pre ďalšie počítače tej doby. Na operačnom systéme Windows je možné hrať hru Had od roku 1992.

\subsection{Zadanie projektu}
Zadaním projektu bolo vytvoriť zjednodušenú hru had na platforme \texttt{FITkit3} a maticovom displeji. Had má pevnú dĺžku a po hracej ploche bude len prechádzať a nebude zbierať žiadne ovocie. Zobrazovanie hada na maticovom displeji má byť riešené pomocou tzv. \texttt{multiplexingu}. Hra sa ovláda pomocou štyroch tlačidiel, ktorými je možné ovládať smer akým sa had pohybuje.



\subsection{Technické detaily}
Program je určený na platformu \texttt{FITkit3} a rozširujúci modul \texttt{Maticový displej}. Program bol vytvorený s využitím vývojového prostredia \texttt{Kinetis Design Studio (KDS)}. 
Implementačným jazykom je \textbf{jazyk~C}.

\begin{figure}[h]
\begin{center}
\includegraphics[scale=0.65]{fitkit.png}
\caption{FITkit3 s pripojeným LED maticovým displejom}
\end{center}
\end{figure}
\section{Implementácia}
\subsection{Programová časť}
Samotný had je reprezentovaný poľom štruktúr \texttt{snake\_cell\_position}, ktoré obsahuje súradnice x,y jednotlivých buniek hada. Ako prvá je zavolaná funkcia \texttt{SystemConfig}, ktorá sa stará o konfiguráciu potrebných súčastí ako napríklad PTA pinov, ktoré slúžia na výber stĺpcov a riadkov na displeji alebo PTE pinov, ktoré sú potrebné na ovládanie pomocou tlačítok. Ďalej sa zavolá funkcia \texttt{PIT\_init} slúžiaca na konfiguráciu PIT časovača. Na počiatočnú inicializáciu hada slúži funkcia \texttt{snake\_init}. Hlavnou riadiacou sučasťou programu sú dva handlery \texttt{PIT0\_IRQHandler} a \texttt{PORTE\_IRQHandler}. Prvý z nich, pre časovač PIT, zaisťuje prerušenia pri pohybe hada a druhý handler pre port E sa stará o stláčanie tlačídiel a zmenu smeru hada nastavením premennej \texttt{direction}. O samotý pohyb hada po hracej ploche sa stará funkcia \texttt{move}, ktorá na základe premennej \texttt{direction} zaisťuje pohyb hada v danom smere. Na výber stĺpca a riadka na maticovom displeji slúžia funkcie \texttt{column\_select} a \texttt{row\_select}.

\subsection{Prevzaté funkcie}
Pri implementácií boli funkcie \texttt{SystemConfig}, \texttt{column\_select} a spomaľovacia funkcia \texttt{delay} prevzaté z demonštračného progamu k tomuto projektu.

\section{Testovanie}
Testovanie prebiehalo manuálne stláčaním tlačítiek na platforme a sledovaním LED maticového displeja. 
\begin{figure}[h]
\begin{center}
\includegraphics[scale=0.05]{had.jpg}
\caption{Testovanie pohybu hada po hracej ploche}
\end{center}
\end{figure}
\pagebreak
\section{Zoznam použitej literatúry}
\verb|[1]| Růžička Richard: Čítače a časovače v mikrokontrolérech, Text k prednáškam kurzu Mikroprocesorové a vestavěné systémy na FIT VUT v Brně. [online], 20. september 2021,[vid. 2021-12-14].  \\\\
\verb|[2]| Růžička Richard: Jádro mikrokontroléru
v mikrokontrolérech, Text k prednáškam kurzu Mikroprocesorové a vestavěné systémy na FIT VUT v Brně. [online], 20. september 2021,[vid. 2021-12-12].
\end{document}